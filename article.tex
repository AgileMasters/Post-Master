\documentclass[lnbip]{svmultln}
\usepackage[utf8]{inputenc}
\usepackage[norsk, english]{babel}
\usepackage[toc,page]{appendix}
\usepackage{cite}
\usepackage{hyperref}
\usepackage{color}
\usepackage{rotating}
\usepackage{adjustbox}
\usepackage{caption}
\usepackage{xcolor}
\usepackage{afterpage}
\usepackage{multirow}
\usepackage{enumitem}
\usepackage{graphicx}
\usepackage[colorinlistoftodos]{todonotes}

\begin{document}

\begin{abstract}
\end{abstract}

\section{introduction}
With the increasing demand of expected delivery from customers, many software development teams adopted agile methodologies as a development practice. With the introduction of these methodologies the practice ``Retrospective'' was introduced. It's roots based in reflective practices from organizational learning such as post-project reviews, experience reports and after action reviews. The purpose of the retrospective is allowing teams to look back at their last development phase and learn from it. Learning from the last period of time could provide useful knowledge that could be used to create improvement opportunities for future iterations or projects.

There have been several studies into the retrospective practice. Dingsøyr, Moe and Nytrø compared light-weight postmortem reviews against experience reports and found that the two provides different kind of experience. light-weight postmortem focuses on implementation, administration, developers and maintence, while experience reports yield experience related to contract issues, design and technology.

On the subject of retrospectives techniques several publications have been done. Stålhane, Dingsøyr, Hanssen and Moe\cite{Hanssen2003} compared two methods of semi-structured interviews and KJ-session and found that both may be applied to harvest knowledge. Bjarnason and Regnell\cite{Bjarnason2012} proposed the evidence-based timelines method as a way of conducting retrospectives. Bjørnson, Wang and Arisholm\cite{bjornson2009} compared the effectiveness of root-cause analysis against that of fishbone diagrams and found that root-cause analysis was more effective as the fishbone diagram limits the way issue were related to each other. Dingsøyr\cite{Dingsoyr2004} have written an article on the subject highlighting the purpose of the retrospective as well as discussing three possible approaches. Dølvik and Stålesen\cite{Dolvik2014} did a literature review on agile retrospectives and found that different techniques would fit different goals, and they found a lack of research considering follow-up on the issues identified during the retrospective. There also several experience reports on the subject, two of them Maham\cite{Maham2008}, Kinoshita\cite{Kinoshita2008}, which describes how to plan and facilitate retrospectives as well as describing different techniques. Finally Derby and Larsen\cite{Larsen2006} has written a book on the subject, describing the purpose and various techniques on conducting the retrospective.

Drury, Conboy and Power\cite{Drury2012} found in a study on obstacles to decision making in agile development teams that the SCRUM-master prioritizes other tasks than follow-up on actions from retrospectives. They proposed that discussion on decision-making should be a part of the retrospective. And they found that some regarded the retrospective as a waste of time while others found it useful. Comparing to this Stålhane, Dingsøyr, Hanssen and Moe\cite{Hanssen2003} found that the participants of their case-study regarded the postmortem as useful.

Zedtwitz\cite{Zedtwitz2002} identified several barriers for learning in post-project reviews in R\&D. Two psychological barriers were identified; inability to reflect and memory bias. Managerial barriers were time constraints and bureaucratic overhead. Epistemological barriers such as difficulty to generalize and tacitness of process knowledge were also identified by Zedtwitz. The last barriers were team-based and identified as reluctance to blame and poor internal communication.

\subsection{Retrospective Definitions}
There are several different definitions of retrospectives. Dingsøyr \cite{Dingsoyr2004} defines postmortem as a collective learning activity in order to improve future practice through organized reflection. Derby and Larsen \cite{Larsen2006} defines retrospectives in another way: \textit{"A special meeting where the team gathers after completing an increment of work to inspect and adapt their methods and teamwork. Retrospectives enable whole-team learning, act as a catalysts for change, and generate action."}

We define retrospectives as the following:\textit{Retrospectives is a process that aims to facilitate shared learning within a team or an organization after a learning event, and thus create a focus to improve current work practices or teamwork.}

\section{Method}
\subsection{Research Goal}
 This article will investigate the outcome of retrospectives, focusing on the organizational learning that happens through retrospectives and the characteristics of the retrospective. When we say characteristics of the retrospective we investigate the output of conducting retrospectives, the processes used and the impediments that faces the retrospective. We investigate these characteristics in the context of existing academic work. \textit{Our goal for this study is to investigate the outcome returned from the retrospective in terms of organizational learning and retrospective characteristics.} Elaborating on this goal we get two sub-goals:

\begin{enumerate}
	\item What are the main characteristics in current retrospective practices, in terms of outcome, processes and impediments?
	\item How is learning achieved through current retrospective practices, in light of organizational learning theory?
\end{enumerate}

The first sub-goal relates to the characteristics of the retrospective. We throughout this study investigate and describe which characteristics are current in todays retrospective. We will focus on the output, the processes used and the impediments. When we say output we investigate what improvement opportunities are created, which improvements are actually implemented and how the enthusiasm evolves as a result of the retrospective. We also investigate the processes used by practitioners of the retrospective and the impediments that hinder them from returning value.

The second sub-goal investigate the learning achieved through the retrospective practice. We employ the learning theory described by Argyris and Schön \cite{Argyris1996} and compare the governing values of their Model I and Model II against the results we see from our case studies. We also investigate which types of learning, single-, double-, or triple-loop, occur through the retrospective practice.

\subsection{Research Design}
In this study we conducted a multiple-case study to investigate our research goal. It consisted of one depth study and one breadth study. A short summary of the research design can be seen in \autoref{table:research-design}.

Our first case study is a depth study investigating the long-term practice of one retrospective practicing agile development team. This is done through analyzing a set of retrospective reports using tabulations and then reflect on the results of this analysis together with the team.

The second case study is a breadth study investigating the retrospective practices of other teams. This is done by interviewing representatives from different retrospective practicing teams.

Our analysis method consists of two parts. The first part is a descriptive discussion on the results found during the two case studies that focuses on the characteristics of the retrospective practice. The second analysis method is comparing our results against the organizational learning framework created by Argyris and Schön \cite{Argyris1996} specifically the governing values of Model I and Model II.

\begin{table}[!h]
	\begin{center}
	\caption{Research design for this multiple-case study.}
	\label{table:research-design}
	\makebox[\textwidth]{%
		\begin{tabular}{l p{0.7\textwidth}}
		\hline
		\textit{Step} & \textit{Description} \\
		\hline
		\textbf{Case 1} & Depth Study \\
		Content Analysis & Tabulation Analysis of Retrospective Reports \\
		Feedback Sessions & Reflection with team about the results of the content analysis. \\[12pt]
		\textbf{Case 2} & Breadth Study \\
		Interview Sessions & Interview different teams \\[12pt]
		\textbf{Analysis Method} & \\
		Characteristics & Descriptive Discussion on Results \\
		Organizational learning & Compare results against Argyris and Schön's \cite{Argyris1996} governing values. \\
		\hline
		\end{tabular}
	}
\end{center}
\end{table}

\section{results} % (fold)
\label{sec:results}

\subsection{Retrospective Characteristics}
In this section we will first describe some key characteristics found during our depth study. We will then continue describing the output, processes and impediments found in both of our studies.

\subsubsection{Key Characteristics}
From our depth study of team Zulu we found some key numbers and these numbers can also be seen in \autoref{table:key-numbers}.The retrospective reports spanned over a period of five years from August 2009 to November 2014. This amounts to 278 weeks and we are going to refer to week numbers from the first retrospective for the remainder of this report.
\begin{table}[!h]
	\begin{center}
	\caption{Some key numbers from the retrospectives}
	\label{table:key-numbers}
	\makebox[\textwidth]{%
		\begin{tabular}{ l | p{0.5\textwidth}}
		\hline
		Key-value & Value  \\
		\hline
		Retrospective report period & 278 Weeks \\
		Number of total actions & 343 \\
		Number of unresolved actions & 65 \\
		Average actions per week & 1.23 \\
		Average unresolved action per week & 0.23 \\
		\hline
		\end{tabular}
	}
\end{center}
\end{table}

\subsubsection{Learning Types in Depth Study}
In terms of organizational learning each action in team Zulu's retrospective reports could be defined as single-loop, double-loop or undefined. The results yielded from the retrospective analysis showed that single-loop was the most occurring type of organizational learning with 66.4\% of the actions. Double-loop had 27.2\% of actions, and the rest was undefined at 6.4\%. The distribution over the timespan, \autoref{figure:learning-l} of the analysis showed that the three categories were evenly distributed. The active actions were very similar to the total amount of actions and only had some small negligible variances as can be seen in \autoref{table:organizational-learning-results}.

\begin{table}[!h]
	\begin{center}
	\caption{Results from the content analysis regarding the organizational learning nature of the action.}
	\label{table:organizational-learning-results}
	\makebox[\textwidth]{%
		\begin{tabular}{| l | l | l | l | l |}
		\hline
		Category & \multicolumn{2}{|c|}{All Actions} & \multicolumn{2}{|c|}{Active Actions}  \\
		\cline{2-5}
		& Number & Percentage & Number & Percentage \\
		\hline
		Single-loop & 227 & 66.4\% & 41 & 66.1\% \\
		Double-loop & 93 & 27.2\% & 16 & 25.8\% \\
		Undefined & 22 & 6.4\% & 5 & 8.1\% \\
		\hline
		\end{tabular}
	}
	\end{center}
\end{table}


\subsubsection{Enthusiasm}
Our case-studies revealed that enthusiasm both inflicted the retrospective and was affected by it. The results revealed that enthusiasm could create both a positive loop and a negative loop. They also revealed some factors that affected the enthusiasm, being oversight and ownership and trust. We will describe each below.

\paragraph{Positive Loop}
Described by several teams, Alfa, Delta, Echo and Zulu, enthusiasm was able to be a part of a positive loop related to the retrospective. The loop could be described as the following:

\begin{quote}
\textit{If the retrospective produced any implementable actions and those actions were implemented it would produce more enthusiasm for the retrospective practice and therefore increase the chance of future actions actually being implemented.}
\end{quote}

It was emphasized that change was important for the retrospective practice by all the teams where the subject came up. It would produce enthusiasm and help improve the working practices.

\paragraph{Negative Loop}
As two sides of a coin enthusiasm could, in addition to create a positive loop, create a negative loop. The SCRUM master of team Charlie told us that unresolved actions could create a negative loop, where enthusiasm for the retrospective went down as improvements never came and the low enthusiasm made sure fewer actions were implemented. And that it could be a challenge.

\paragraph{Oversight}
<<<<<<< HEAD
From the second feedback session with team Zulu during the depth analysis we learned that oversight over implementation rate of actions had affected the retrospective. The team had not previously appreciated just how much had been accomplished, and that this had lead to a lower enthusiasm around the retrospective. When the team was confronted with the completion rate they were surprised and pleased with the amount they had accomplished. In order to continue this more objective oversight the team decided to create a dashboard in order to have better view of the statistics concerning implemented versus unimplemented actions as described in \autoref{dashboard}.
=======
From the second feedback session with team Zulu during the depth analysis we learned that oversight over implementation rate of actions had affected the retrospective. The team had not previously appreciated just how much had been accomplished, and that this had lead to a lower enthusiasm around the retrospective. When the team was confronted with the completion rate they were surprised and pleased with the amount they had accomplished.
>>>>>>> 10961c57daa553890ad8a9b2dd2cf838e8cb9d03


\paragraph{External Facilitator}
There were different views on facilitating retrospectives. Several teams used an external facilitator and said that they would encourage others to do the same. The benefits was that the external facilitator was able to see things that existed within the team, that the team themselves were not aware of. The external facilitator was not hired as a facilitator, but rather a SCRUM master from another development team.

<<<<<<< HEAD
The use of external facilitators was an interesting concept to team Zulu, and the one experience they had with using one had been a positive experience. When the team Zulu SCRUM master was asked about if he felt like a leader, or if the team viewed him as a leader, the SCRUM master said he felt more like a facilitator, and that he didn't think the team considered him a leader. Though he was mindful of this possibility. When asked about the inclusion of the project leader he considered as long as the team was not afraid to speak their minds this could make it easier to make strategic decisions.

Other teams had not been using external facilitators. They used their regular SCRUM master. Common for the SCRUM masters were that they all felt like they were a facilitator rather than a leader during the retrospectives. Those we spoke to about external facilitators were positive to the idea and mentioned they might want to try it out in the future.
=======
Other teams had not been using external facilitators. They used their regular SCRUM master. Common for the SCRUM masters were that they all felt like they were a facilitator rather than a leader during the retrospectives. Those we spoke to about external facilitators were positive to the idea and mentioned they might want to try it out in the future.
>>>>>>> 10961c57daa553890ad8a9b2dd2cf838e8cb9d03

\begin{table}[!h]
	\begin{center}
	\caption{Usage of external facilitator}
	\label{table:external-facilitator}
	\makebox[\textwidth]{%
		\begin{tabular}{ l | p{0.3\textwidth}}
			External facilitator & Teams \\
			\hline
			Used external facilitator & Alfa, Echo \\
			No external facilitator & Charlie, Delta, Foxtrot, Golf, Bravo \\
		\end{tabular}
	}
	\end{center}
\end{table}

\subsubsection{Implementation}
\label{question-6}
The steps taken to implement actions created during the retrospective and whether unresolved actions were a problem varied between the interviewed teams.

\begin{table}[!h]
	\begin{center}
	\caption{Action Follow-Up Techniques Used}
	\label{table:follow-up-techique}
	\makebox[\textwidth]{%
		\begin{tabular}{ l | l | l }
		Team & Follow-up technique & Satisfied \\
		\hline
		Echo, Delta, Alfa, Foxtrot, Zulu & Assigning Name to action & Yes \\
		Charlie, Bravo & Assigning to group & No \\
		Foxtrot & Adding to backlog & No \\
		Delta & Visualizes Action & Yes \\
		Echo, Golf & Handled by SCRUM-master & Yes \\
		\hline
		\end{tabular}
	}
	\end{center}
\end{table}


% section discussion (end)

\section{Discussion}

\subsection{Retrospective Characteristics}
One of our research objectives was: \textit{What are the main characteristics in current retrospective practices, in
terms of outcome, processes and impediments?} Throughout this section we will provide a descriptive discussion on the characteristics we have identified throughout our studies. The section is split into three parts: Output, Process Characteristics and Impediments. An overview of all the characteristics can be seen in \autoref{table:retrospective-properties}

\begin{table}[h]
	\begin{center}
		\caption{Retrospective characteristics}
		\label{table:retrospective-properties}
		\begin{tabular}{p{0.9\textwidth}}
			\hline
			\textit{Retrospective Characteristics}\\
			\hline
			\textbf{Output} \\
			Reflection on work processes, technical issues, work environment \\
			Creates improvement opportunities \\
			Improvement implementation \\
			Provides organizational learning \\
			Can improve team enthusiasm \\
			Can decrease team enthusiasm \\ 
			Can improve efficiency  \\
			Facilitates empowerment \\
			\hline 
			\textbf{Process Characteristics}\\
			Little considerations taken \\
			Wish to improve \\
			Varying techniques\\
			Occurs regularly\\
			Collects opinions from participants\\
			Arena for open discussion \\
			Allows for experiments in work environment\\
			Shared learning event\\
			\hline
			\textbf{Impediments}\\
			Team commitment\\
			Enforcing of process improvement actions\\
			\hline
		\end{tabular}
	\end{center}
\end{table}

\subsubsection{Output}

\paragraph{Creates Improvement Opportunities}
Through our studies we have seen that teams are able to improve based on decisions created during the retrospective. From the depth study of Team Zulu we saw that through 77 retrospectives the team had created 343 actions which reflect improvement opportunities. Also all the teams in our breadth study created actions to improve some aspect of their work-life. It is clearly evidence that improvement opportunities are created through the retrospective. This seems to fit well with previous research \cite{Larsen2006, Dingsoyr2004, Drury2012} that the retrospective help identify improvement opportunities. 

\paragraph{Improvement Implementation}
The question if the improvement opportunities are actually implemented can also be seen through our studies. The results of our studies revealed that most of the teams were satisfied with their implementation rate. From team Zulu we learned that only 65 of the 343 actions were not yet implemented. It was also revealed that implementation could be a challenge, as was the case with team Charlie. We identified two methods that seemed to help overcome this challenge. The first was assigning responsible team members to each action. The second was SCRUM master follow-up of the actions. We have seen that retrospective practicing agile development teams are able to implement improvement opportunities confirming Derby and Larsen's \cite{Larsen2006} statement of retrospective helping team adapt. This is contradicting the research of Drury et. al. \cite{Drury2012} which finds that no real changes occur as a result of the retrospective. 

\paragraph{Can Improve Efficiency}
The retrospective practice can, as we have seen in multiple examples, improve the efficiency of teams conducting them. Team Echo's practice changes from SCRUM to KANBAN and then to modified SCRUM and Team Zulu's ``Bug-crunch day'' are just two of the examples we have seen that the teams have been able to improve their efficiency through the retrospective practice. This again confirms the previous literature \cite{Dingsoyr2004, Larsen2006, Kinoshita2008} that retrospectives are able to improve practices and contradicts the finding of Drury et. al. \cite{Drury2012} that retrospectives provide no real changes.

\paragraph{Allows for Experiments in the Work Place}
\label{section:experiments-in-work-place}
As seen in \autoref{section:Derby-Larsen-Structure} a retrospective can be an area that facilitates experimentation in a team. We have observed this, for example as described with team Echo in \autoref{question-11}. Here team Echo tried to move to KANBAN from SCRUM, the experiment was not an immediate success, but allowed them to return to a SCRUM methodology that they could tailor after their experiences from the experiment. This resulted in their work methodology fitting their team better. Our work with team Zulu also led to experimentation, for example their decision to create a dashboard to log their retrospective actions. Another experiment by team Zulu was the inclusion of a ``bug-crunch'' day seen in \autoref{question-11}. This experiment was a success and led to a noticeable decrease in bugs.

\paragraph{Shared Learning Event}
The theory of the retrospective as a shared learning event was described in \autoref{intro:retro-outcome}.  An example of a shared learning event from the same section was performed by team Delta, as they changed their time estimation practices to great success after discussing the process in a retrospective. However none of the teams interviewed made an organized effort to make the result of the retrospective a learning event for personnel outside the team, as recommended by Dingsøyr ~\cite{Dingsoyr2004}.

\paragraph{Enthusiasm}
\label{section:positive-loop-enthusiasm}
Team enthusiasm is both affected by the retrospective practice and inflicted by it. It can be increased through a positive feedback loop or decreased by a negative feedback loop. As a result of the retrospective practice individual empowerment is facilitated and this also increases the enthusiasm. We will describe and discuss each of these statements below through examples from our studies and earlier literature. 
\label{posi-loop}
We have seen that the enthusiasm of the participants of retrospective practice can be both affected by the retrospective. We uncovered a positive loop that helps increase the enthusiasm of the team conducting retrospectives. If changes occurred as a result of the retrospective the participants would become enthusiastic and thus the chance of more changes would occur. Ownership towards the development process was one factor that could help increase the enthusiasm and feed the positive loop. This positive loop confirms what Derby and Larsen \cite{Larsen2006} states that teams are invested in the success of improving their work as the improvements are chosen by the teams themselves and not from upper management. 
\label{negi-loop}
The retrospective practice has also the ability to decrease the enthusiasm of the practicing team. As the opposite of the positive loop a negative are able to decrease the enthusiasm. If no changes occur enthusiasm will decrease and as the enthusiasm decreases the chance of new changes occurring decreases. Some even might see the retrospective as a waste of time as described by team Charlie's SCRUM master which had happened with some of the teams in his department. This confirms our previous literature review \cite{Dolvik2014} that recurring issues kill the joy and Drury et. al. \cite{Drury2012} research that some may see the retrospective as a waste of time. 

\paragraph{Facilitates Empowerment}
Ownership towards the development process was seen as crucial towards getting improvement out of the retrospective by the interviewed teams. Each team member has the possibility to participate in shaping their working process through the retrospective. As seen in team Alfa and Echo even the shy are required to participate in returning feedback and contribute solutions for current work processes. Tessem \cite{Tessem2014} identifies participation in process improvements as an empowering practice. This directly relates to the retrospective practice, which is also a parallel drawn by Tessem and our work supports this. Enthusiasm is increased as members are empowered\cite{Tessem2014} and thus the retrospective increase enthusiasm through empowerment. 

\subsubsection{Process Characteristics }

When we consider our results we see that few of the teams interviewed do a thorough consideration of their practices in relation to the approaches seen in \autoref{table:postmortem-approach}. Most teams did not do a informed decision on several of Dingsøyr's considerations. For example on who to invite or sharing tacit and explicit knowledge. For example team Alpha's interviewee said that it could go long periods of time where a developer was not invited to a retrospective. When it comes to sharing the knowledge generated none of the teams made a concerted effort to share the knowledge further in their organization that came as a results of learning through the retrospective.

Many teams had a great wish to improve. As seen throughout our results the need to build a culture that allows for learning is absolutely essential for a productive environment. This is in accord with  Dingsøyr's work. Especially trust between the team members emerged as critical for reaching the maximum potential of a retrospective session. An example of the possible improvements is the team dynamic improvements experienced by team Delta when one of their team members brought up the problem that he was not getting help from other team members, as described in \autoref{question-21}. Also after our depth analysis with team Zulu their eagerness to improve let them turn the results from our analysis into a basis for multiple actions intended to improve the team's learning capabilities. 


\subsubsection{Impediments}

Some personalities could provide obstacles for the retrospective. We saw three examples of this. The first one was in team Zulu where cultural differences provided miscommunication and difficulty providing an open discussion in the retrospective. The second was in the department to team Charlie where some SCRUM masters had low enthusiasm for the retrospective practice and this could influence the rest of the teams as well. The third example were two senior developers with strong personalities, in team Charlie, that could hinder the other developers from voicing their opinions. These were the only three cases we saw in our studies, but further personalities may be investigated. Derby and Larsen \cite{Larsen2006} talks about personalities that take a lot of time and hinder others from taking part in the retrospective and this is reflected pretty well in the last example.

Even though most of the teams in our study was satisfied with the commitment from their teams, implementation of actions still could provide a challenge. As mentioned by several of the interviewees if the actions were not assigned to a specific person the action would not be implemented. This indicates that the team as a whole don't have the commitment to implementation of retrospective actions. Drury et. al. \cite{Drury2012} described several obstacles that fit with these results. Team members unwilling to commit and relying on SCRUM master, not implementing decisions or relying on others for decisions, and not taking ownership of decisions are all obstacles that could be identified through our research. However as mentioned this was only the case when the group as whole was assigned to a decision, when individuals of the group were assigned the teams were satisfied with the implementation rate. This still indicates however a lack of team commitment even though the consequences are dealt with. 

Enforcing the implementation of process improvement actions was seen as a challenge. The SCRUM master of team Zulu described how enforcing and monitoring process improvement actions was a challenge. Some process changes required the whole team to implement the action and both enforcing the implementation and monitor it could be hard. Conflicting priorities and not taking ownership for decisions are two of Drury's et. al. \cite{Drury2012} obstacles to effective decision making, and these results reflects the two obstacles. 

The last impediment we have seen for conducting retrospective is availability. In team Golf we saw how having a distributed team could make it harder to conduct the retrospective. In team Alfa and Foxtrot we learned how the unavailability of retrospective due to long timespan or participants not being able to participate could inhibit the retrospective. Zedtwitz's \cite{Zedtwitz2002} barrier of memory bias and Drury et. al. \cite{Drury2012} obstacle of unavailable staff is reflected in this. 

\subsection{Organization Learning}
One of our research objectives were: \textit{How is learning achieved through current retrospective practices, in light of organizational learning theory?} Throughout this section we will first discuss the results of our case-studies in terms of the governing values of Argyris and Schön's \cite{Argyris1996} organizational learning Model I and Model II. Secondly we will discuss our results in terms of learning types. 

\subsubsection{Model I}
In Model I, described in \autoref{sub:model_i}, four governing values set the focus for the learning organization. In our investigation of the retrospective practice we have seen very little to any of these values. We will discuss this below for each governing value and then the consequences before we summarize the findings related to Model I.  

\paragraph{Setting and  Achieving Goals}
One could argue that the team setting goals and achieving them could be compared to creating actions and fulfilling them, however we do not support this as the fulfillment of actions is part of a collective efficiency improvement and learning practice. The retrospective and its participants are instead of trying to design and manage the environment unilaterally, investigating all the different angles and perspectives the team can present and finding solutions to the problems existing within that environment. The joint team discussion and retrospective practices are evidence of this. 
\section{Conclusion}
Throughout this thesis we have conducted an empirical study, of mature agile development teams. Investigating the outcome returned from the retrospective in terms of organizational learning and retrospective characteristics. Thereby answering Dingsøyr and Dybå’s call \cite{Dyba2008} for empirical studies into mature agile development teams. For the practitioners we have investigated the outcome of the characteristics and proposed a set of guidelines which could help improve their practice.

The current characteristics of todays retrospective we have seen that the outcome of the practice is improvement opportunities and learning which could result in improved efficiency, increased enthusiasm and adaptation of work-processes, work environment, and product quality. For the processes of the practice we have identified a feedback loop with a barrier for implementation of improvement opportunities, depending on team commitment for implementation and enthusiasm for the retrospective. 

\begin{itemize}
\item Retrospectives allow teams to improve their current work-practices through learning, team commitment and investigation of past development phases.
\end{itemize}

The studies revealed that mature agile development teams that practice retrospectives is approximating an organizational learning II system as described by Argyris and Schön \cite{Argyris1996}. We have identified one barrier to double-loop learning which consists of several factors. We have seen that reflection on ones own learning helps give focus to the retrospective and lower this barrier and we have proposed a method for this. 

\begin{itemize}
\item Learning in the present retrospective practices are primary single-loop learning, however the learning environment could facilitate double-loop learning.
\end{itemize}

Previous critique \cite{Drury2012} have stated that the retrospective does not provide any changes to the work environment and we have observed that this could be the case if the team are not able to overcome the team commitment barrier for implementation of improvement opportunities.  

Finally todays retrospective practices provide agile development teams the ability to adapt their current work-practices and enables them to learn from past development iterations and thus provide the means for identifying improvement opportunities and improve from them. 

\section{Future Work}
For future work we would recommend conducting similar studies to increase the sample size of this research. Further development to the meta-retrospective and researh into the effects of it would also be intersting to see. Determining the value of doing double-loop learning and investigating the retrospectives through models of shared mental models would also help enrich the resarch of knowledge sharing and the practice of retrospectives. 


\bibliography{references}
\bibliographystyle{plain}
\end{document}
