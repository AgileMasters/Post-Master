\section{introduction}
With the increasing demand of expected delivery from customers, many software development teams adopted agile methodologies as a development practice. With the introduction of these methodologies a practice called ``Retrospective'' was introduced. It's roots based in reflective practices from organizational learning such as post-project reviews, experience reports and after action reviews. The purpose of the retrospective is allowing teams to look back at their last development phase and learn from it. Learning from the last period of time could provide useful knowledge that could be used to create improvement opportunities for future phases or projects. 


\subsection{Retrospective Definitions}
There are several different definitions of retrospectives. Dingsøyr \cite{Dingsoyr2004} defines postmortem as a collective learning activity in order to improve future practice through organized reflection. Derby and Larsen \cite{Larsen2006} defines retrospectives in another way: \textit{"A special meeting where the team gathers after completing an increment of work to inspect and adapt their methods and teamwork. Retrospectives enable whole-team learning, act as a catalysts for change, and generate action."}

We define retrospectives as the following:\textit{Retrospectives is a process that aims to facilitate shared learning within a team or an organization after a learning event, and thus create a focus to improve current work practices or teamwork.}


\subsection{Retrospectives: Earlier Academic Work}
Dingsøyr, Moe and Nytrø compared light-weight postmortem reviews against experience reports and found that the two provides different kind of experience. light-weight postmortem focuses on implementation, administration, developers and maintence, while experience reports yield experience related to contract issues, design and technology. 

On the subject of retrospectives techniques several publications have been done. Stålhane, Dingsøyr, Hanssen and Moe\cite{Hanssen2003} compared two methods of semi-structured interviews and KJ-session and found that both may be applied to harvest knowledge. Bjarnason and Regnell\cite{Bjarnason2012} proposed the evidence-based timelines method as a way of conducting retrospectives. Bjørnson, Wang and Arisholm\cite{bjornson2009} compared the effectiveness of root-cause analysis against that of fishbone diagrams and found that root-cause analysis was more effective as the fishbone diagram limits the way issue were related to each other. Dingsøyr\cite{Dingsoyr2004} have written an article on the subject highlighting the purpose of the retrospective as well as discussing three possible approaches. Dølvik and Stålesen\cite{Dolvik2014} did a literature review on agile retrospectives and found that different techniques would fit different goals, and they found a lack of research considering follow-up on the issues identified during the retrospective. There also several experience reports on the subject, two of them Maham\cite{Maham2008}, Kinoshita\cite{Kinoshita2008}, which describes how to plan and facilitate retrospectives as well as describing different techniques. Finally Derby and Larsen\cite{Larsen2006} has as previously mentioned a book on the subject. 

Other work returns some different findings. Drury, Conboy and Power\cite{Drury2012} found in a study on obstacles to decision making in agile development teams that the SCRUM-master prioritizes other tasks than follow-up on actions from retrospectives. They proposed that discussion on decision-making should be a part of the retrospective. And they found that some regarded the retrospective as a waste of time while others found it useful. Comparing to this Stålhane, Dingsøyr, Hanssen and Moe\cite{Hanssen2003} found that the participants of their case-study regarded the postmortem as useful. 

Zedtwitz\cite{Zedtwitz2002} identified several barriers for learning in post-project reviews in R\&D. Two psychological barriers were identified; inability to reflect and memory bias. Managerial barriers were time constraints and bureaucratic overhead. Epistemological barriers such as difficulty to generalize and tacitness of process knowledge were also identified by Zedtwitz. The last barriers were team-based and identified as reluctance to blame and poor internal communication. 
